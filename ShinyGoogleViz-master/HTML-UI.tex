\documentclass{beamer}

\usepackage{default}

\begin{document}
%--------------------------------------------------------------------------------------------------%

\begin{frame}
The HTML UI application demonstrates defining a Shiny user-interface using a standard HTML page rather than a ui.R script. To run the example type:
\begin{verbatim}
> library(shiny)
> runExample("08_html")
\end{verbatim}
\end{frame}
%--------------------------------------------------------------------------------------------------%

\begin{frame}
\frametitle{Defining an HTML UI}
The previous examples in this tutorial used a \texttt{ui.R} file to build their user-interfaces. While this is a fast and convenient way to build user-interfaces, some appliations will inevitably require more flexiblity. For this type of application, you can define your user-interface directly in HTML. In this case there is no \texttt{ui.R} file and the directory structure looks like this:
\end{frame}
%--------------------------------------------------------------------------------------------------%
\begin{frame}


<application-dir>
|-- www
    |-- index.html
|-- server.R
In this example we re-write the front-end of the Tabsets application using HTML directly. Here is the source code for the new user-interface definition:

www/index.html
<html>

<head>
  <script src="shared/jquery.js" type="text/javascript"></script>
  <script src="shared/shiny.js" type="text/javascript"></script>
  <link rel="stylesheet" type="text/css" href="shared/shiny.css"/> 
</head>
 
<body>
  <h1>HTML UI</h1>
 
  <p>
    <label>Distribution type:</label><br />
    <select name="dist">
      <option value="norm">Normal</option>
      <option value="unif">Uniform</option>
      <option value="lnorm">Log-normal</option>
      <option value="exp">Exponential</option>
    </select> 
  </p>
 
  <p>
    <label>Number of observations:</label><br /> 
    <input type="number" name="n" value="500" min="1" max="1000" />
  </p>
 
  <pre id="summary" class="shiny-text-output"></pre> 
  
  <div id="plot" class="shiny-plot-output" 
       style="width: 100%; height: 400px"></div> 
  
  <div id="table" class="shiny-html-output"></div>
</body>

</html>
\end{verbatim}
\end{frame}
\begin{frame}
There are few things to point out regarding how Shiny binds HTML elements back to inputs and outputs:

HTML form elmements (in this case a select list and a number input) are bound to input slots using their name attribute.
Output is rendered into HTML elements based on matching their id attribute to an output slot and by specifying the requisite css class for the element (in this case either shiny-text-output, shiny-plot-output, or shiny-html-output).
With this technique you can create highly customized user-interfaces using whatever HTML, CSS, and JavaScript you like.
\end{frame}
%---------------------------------------------------------------------%
\begin{frame}
Server Script
All of the changes from the original Tabsets application were to the user-interface, the server script remains the same:

server.R
library(shiny)

# Define server logic for random distribution application
shinyServer(function(input, output) {

  # Reactive expression to generate the requested distribution. This is 
  # called whenever the inputs change. The output renderers defined 
  # below then all used the value computed from this expression
  data <- reactive({  
    dist <- switch(input$dist,
                   norm = rnorm,
                   unif = runif,
                   lnorm = rlnorm,
                   exp = rexp,
                   rnorm)

    dist(input$n)
  })

  # Generate a plot of the data. Also uses the inputs to build the 
  # plot label. Note that the dependencies on both the inputs and
  # the data reactive expression are both tracked, and all expressions 
  # are called in the sequence implied by the dependency graph
  output$plot <- renderPlot({
    dist <- input$dist
    n <- input$n

    hist(data(), 
         main=paste('r', dist, '(', n, ')', sep=''))
  })

  # Generate a summary of the data
  output$summary <- renderPrint({
    summary(data())
  })

  # Generate an HTML table view of the data
  output$table <- renderTable({
    data.frame(x=data())
  })
})
\end{frame}
%---------------------------------------------------------------------%

\end{document}
