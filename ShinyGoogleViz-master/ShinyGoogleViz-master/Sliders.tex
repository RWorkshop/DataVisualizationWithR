The Sliders application demonstrates the many capabilities of slider controls, including the ability to run an animation sequence. To run the example type:

> library(shiny)
> runExample("05_sliders")
%---------------------------------------------------------------------------------%
Customizing Sliders
Shiny slider controls are extremely capable and customizable. Features supported include:

The ability to input both single values and ranges
Custom formats for value display (e.g for currency)
The ability to animate the slider across a range of values
Slider controls are created by calling the sliderInput function. The ui.R file demonstrates using sliders with a variety of options:
%---------------------------------------------------------------------------------%
ui.R
library(shiny)

# Define UI for slider demo application
shinyUI(pageWithSidebar(

  #  Application title
  headerPanel("Sliders"),

  # Sidebar with sliders that demonstrate various available options
  sidebarPanel(
    # Simple integer interval
    sliderInput("integer", "Integer:", 
                min=0, max=1000, value=500),

    # Decimal interval with step value
    sliderInput("decimal", "Decimal:", 
                min = 0, max = 1, value = 0.5, step= 0.1),

    # Specification of range within an interval
    sliderInput("range", "Range:",
                min = 1, max = 1000, value = c(200,500)),

    # Provide a custom currency format for value display, with basic animation
    sliderInput("format", "Custom Format:", 
                min = 0, max = 10000, value = 0, step = 2500,
                format="$#,##0", locale="us", animate=TRUE),

    # Animation with custom interval (in ms) to control speed, plus looping
    sliderInput("animation", "Looping Animation:", 1, 2000, 1, step = 10, 
                animate=animationOptions(interval=300, loop=T))
  ),

  # Show a table summarizing the values entered
  mainPanel(
    tableOutput("values")
  )
))
%---------------------------------------------------------------------------------%
Server Script
The server side of the Slider application is very straightforward: it creates a data frame containing all of the input values and then renders it as an HTML table:

server.R
library(shiny)

# Define server logic for slider examples
shinyServer(function(input, output) {

  # Reactive expression to compose a data frame containing all of the values
  sliderValues <- reactive({

    # Compose data frame
    data.frame(
      Name = c("Integer", 
               "Decimal",
               "Range",
               "Custom Format",
               "Animation"),
      Value = as.character(c(input$integer, 
                             input$decimal,
                             paste(input$range, collapse=' '),
                             input$format,
                             input$animation)), 
      stringsAsFactors=FALSE)
  }) 

  # Show the values using an HTML table
  output$values <- renderTable({
    sliderValues()
  })
})
%---------------------------------------------------------------------------------%
