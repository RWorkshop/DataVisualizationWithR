
%============================================%

\begin{frame} 
\frametitle{ Data Visualization with ggvis}
\begin{itemize}

\item piping with the magrittr package

\item

\end{itemize}

\end{frame}

%============================================%

\begin{frame} 

\frametitle{ Data Visualization with ggvis}
\begin{itemize}

\item

\item

\end{itemize}

\end{frame}

%============================================%
http://cdn.oreillystatic.com/en/assets/1/event/120/ggvis_%20Interactive,%20intuitive%20graphics%20in%20R%20Presentation.pdf

You will also need to open RStudio and run the following command at the
command prompt before running the demos:
install.packages(c("dplyr", "ggvis", "knitr",
 "lubridate", "nycflights", "shiny", "tourr"))

%===================================================%

2Animated rotation
© 2014 RStudio, Inc. All rights reserved.
slides at: bit.ly/ggvis-barcelona
# rotation.R
library(tourr)
library(ggvis)
library(shiny)
aps <- 2
fps <- 30
mat <- rescale(as.matrix(flea[c(1,3,4)]))
tour <- new_tour(mat, grand_tour(), NULL)
start <- tour(0)
proj_data <- reactive({
 invalidateLater(1000 / fps, NULL);
 step <- tour(aps / fps)
 data.frame(center(mat %*% step$proj), species = flea$species)
})
proj_data %>% ggvis(~X1, ~X2, fill = ~species) %>%
 layer_points() %>%
 scale_numeric("x", domain = c(-1, 1)) %>%
 scale_numeric("y", domain = c(-1, 1)) %>%
 set_options(duration = 0)



%%- http://ggvis.rstudio.com/

Interactive, web-ready ggplot2-style graphics with ggvis
Hadley Wickham's been working on the next-generation update to ggplot2 for a while, and now it's available on CRAN. The ggvis package is completely new, and combines a chaining syntax reminiscent of dplyr with the grammar of graphics concepts of ggplot2. The resulting charts are web-ready in scalable SVG format, and can easily be made interactive thanks to RStudio's shiny package.

For example, here's the code to create a scatterplot with a smoothing line from the mtcars data set:

mtcars %>% 
  ggvis(~wt, ~mpg) %>%
  layer_points() %>%
  layer_smooths()
And here's the corresponding SVG image:



SVG graphics are great online, because they're compact (this one's just 25Kb) and look great whatever size they're displayed at (it's a vector format, so you never get pixellation). The only think SVGs don't work well for is charts with millions of elements (points, lines, etc.) because then they can be large and slow to render. (The only other downside is that our blogging platform, TypePad, doesn't support SVG with its image tools, so I had to insert an <image> element into the HTML directly.)

You can easily add interactivity to a chart, by specifying parameters as input controls rather than numbers. Here's the code for the same chart, with a slider to specify the smoothing parameter and point size:

mtcars %>%
  ggvis(~wt, ~mpg) %>%
  layer_smooths(span = input_slider(0.5, 1, value = 1)) %>%
  layer_points(size := input_slider(100, 1000, value = 100))
If you run that code in RStudio you'll get an interactive chart, or go here to see the same interactivity on a web page, rendered with RStudio's Shiny. For more details, check out the ggvis website linked below.
%================================================================================================================================================================%

%=========================================================================================================%

Examples of ggvis graphics
Histogram:

1.5
2.0
2.5
3.0
3.5
4.0
4.5
5.0
eruptions
0
5
10
15
20
25
30
35
count
Scatterplot with smooth curve and interactive control:

%================================================================%
ggvis vs ggplot2
If you’re familiar with ggplot2, learning ggvis shouldn’t be too hard - it borrows from many familiar concepts. Note that ggvis is still very young, and many of the interfaces are likely to change as we learn more about what works well.

Basic naming conversions:
layer, geom -> layer function
stat -> compute function
aes() -> props()
ggplot() -> ggvis()
ggvis has a function interface so you combine components using %>%, not + as in ggplot2.

Facetting is not currently supported, and when it is supported, it’s more like to resemble embedded plots than facetting in ggplot2.

In ggplot2, the definition of a geom was somewhat blurred, because of things like geom_histogram() which combined geom_bar() with stat_bin(). The distinction is more clear in ggvis: pure geoms correspond to emit_* which emit marks, and combined geoms and stats correspond to layers.

Using ggvis() without adding any layers is analogous to qplot().

Vega provides a smaller set of scales than ggplot2 (just ordinal, quantitative, and time), but they are much more flexible than ggplot2 scales, and offer equivalent functionality.

ggplot2 has a two-level hierarchy - you have data and aes specifications in the plot and in each layer. ggvis provides an unlimited hierarchy - you can have as many levels as you need (and the data will only be computed once)

ggvis makes fewer assumptions about the type of data - data does not have to be a data frame until it has been processed by a transform.
