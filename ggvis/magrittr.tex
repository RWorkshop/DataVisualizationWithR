
\documentclass[MASTER.tex]{subfiles} 
\begin{document} 

	%=================================================%
	\begin{frame}
		\begin{figure}
\centering
\includegraphics[width=0.99\linewidth]{images/pipe}

\end{figure}
\begin{center}
		{\huge THE $ \%>\% $ OPERATOR}
\end{center}

		
		
	\end{frame}
	
	\begin{frame}
		\begin{figure}
\centering
\includegraphics[width=0.99\linewidth]{images/pipe2}

\end{figure}

	\end{frame}
	%=================================================%
	\begin{frame}[fragile]
		\frametitle{ The $\%>\%$ operator}
		\LARGE
		% - http://r2014-mtp.sciencesconf.org/file/92631
		\begin{itemize}
		\item From \textbf{magrittr} package. 
		\item Used extensively in \textbf{dplyr}.
		\item $\%>\%$ is a piping operator, and can be verbalised as ``\textit{then}".
		\item It takes the output of the left side, and uses it as the first
		argument of the function on the right side.
		\end{itemize}
		\end{frame}
		%================================================================= %
		\begin{frame}[fragile]
			\frametitle{magrittr :  the $\%>\%$ operator}
			\Large
		\begin{framed}
		\begin{verbatim}
		subset(mtcars, cyl == 6, c(mpg, wt))
		
		mtcars %>% subset(cyl == 6, c(mpg, wt))
		\end{verbatim}
	\end{framed}
	
\end{frame}
		%================================================================= %
		\begin{frame}[fragile]
		\frametitle{magrittr :  the $\%>\%$ operator}
		\Large
		\begin{framed}
		\begin{verbatim}
			
		summary(subset(mtcars, cyl == 6, 
		           c(mpg, wt)), digits=2)
		  
		mtcars %>% 
				   subset(cyl == 6, c(mpg, wt)) %>% 
				   summary(digits=2)
		\end{verbatim}
		\end{framed}
	
	\end{frame}
		%================================================================= %
		\begin{frame}[fragile]
			\frametitle{magrittr :  the $\%>\%$ operator}
			\Large
			\begin{framed}
				\begin{verbatim}	
				mtcars %>% 
				   subset(cyl == 6, c(mpg, wt)) %>% 
				      summary(digits=2)
				\end{verbatim}
			\end{framed}
			\begin{itemize}
				\item Get the mtcars data set
				\item \textbf{Then} subset it like this
				\item \textbf{Then} get the summary, with this setting
			\end{itemize}
			\end{frame}
	%================================================================= %
	\begin{frame}
		\frametitle{magrittr :  the $\%>\%$ operator}
		\begin{itemize}
			\item You can use the $\%>\%$ operator with any \texttt{R} functions.
			\item The rules are simple: the object on the left hand side is passed as the first argument to the function on the right hand side. So: 
		\end{itemize}
		\begin{framed}	
			my.data $\%>\%$ \texttt{my.function} is the same as \texttt{my.function(my.data)}
			my.data $\%>\%$ \texttt{my.function(arg=value)} is the same as \texttt{my.function(my.data, arg=value)}
		\end{framed}
		
	\end{frame}
	%=================================================%
	\begin{frame}[fragile]
		\frametitle{The $ \%>\% $ Operator}
		\Large
		\textbf{The $ \%>\% $ Operator}
		\begin{itemize}
			\item ggvis makes use of the $ \%>\% $ operator from the
			package magrittr
			\item This allows us to layer up graphics in the same
			way we would with ggplot2
			
		\end{itemize}
		
		
	\end{frame}
	%=================================================%
	\begin{frame}[fragile]
		\frametitle{The $ \%>\% $ Operator}
		\textbf{Tube Data Example }\\ (\textit{Dr. Aimee Gott, Mango Solutions})
%		\begin{itemize}
%			\item The $ \%>\% $ operator passes the left hand object to the
%			first argument of the right hand expression
%			\item We can pass data or objects to functions in this way
%			
%		\end{itemize}
		
		\begin{framed}
			\begin{verbatim}
			> tubeData$Excess %>%  tapply(tubeData$Line, mean)
			
			# Bakerloo              5.047714
			# Central               5.998667
			# Circle & HamDistrict  7.166095 
	
			\end{verbatim}
		\end{framed}
		
	\end{frame}
	%================================================================= %
	\begin{frame}
		\frametitle{magrittr :  the $\%>\%$ operator}
		\Large
		\vspace{-1cm}
		\textbf{	$ \%>\% $ in ggvis}
		\begin{itemize}
			\item With ggvis we pass "\texttt{ggvis}" objects
			\item We create the initial object by passing data to
			\texttt{ggvis()}
			\item All other functions expect a ggvis object as the
			first argument and return a ggvis object
		\end{itemize}
		
	\end{frame}
	
	%============================================================================= %
\end{document}
