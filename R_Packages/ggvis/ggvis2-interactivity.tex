
\documentclass[MASTER.tex]{subfiles} 
\begin{document} 
	
%============================================================================== %
\begin{frame}[fragile]
\frametitle{Interactivity}
{\LARGE MAKING PLOTS INTERACTIVE} \\ \bigskip
\large As well as mapping visual properties to variables or setting them to specific values, you can also connect them to interactive controls.

\end{frame}
%=================================================%
\begin{frame}
\frametitle{Basic interactivity}
\large
\textbf{Basic Interactivity}
\begin{itemize}
\item The most basic interactivity we can add is "hover
over" changes
\item We can change properties by using
\texttt{property.hover} arguments
\texttt{fill.hover := "red"}
\end{itemize}
\end{frame}
%=================================================%
\begin{frame}[fragile]
	\frametitle{Basic interactivity}
\begin{framed}
\begin{verbatim}
tubeData %>%
  ggvis(~Excess)  %>%
     layer_histograms(fill.hover = "red") 
\end{verbatim}
\end{framed}
\end{frame}
%=================================================%
\begin{frame}[fragile]
\frametitle{Interactive Input}
\Large
\textbf{The := operator}
\begin{itemize}
	\item We can also set properties to be the output of an
	interactive control
	
	\item We use the setting "\texttt{:=}" for this input
	\item We can optionally set labels next to the control
\end{itemize}
\begin{framed}
\begin{verbatim}
opacity := input_slider(0, 1, label = "Opacity")
\end{verbatim}
\end{framed}
\end{frame}
\end{document}
