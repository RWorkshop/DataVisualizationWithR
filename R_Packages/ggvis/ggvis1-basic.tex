\documentclass[MASTER.tex]{subfiles} 
\begin{document} 

%==========================================================================%
\begin{frame}
\frametitle{Data Visualization with ggvis}
\Large
\textbf{Introduction}
\begin{itemize}
\item The goal of \textbf{ggvis} is to make it easy to build \textbf{interactive graphics} for exploratory data analysis. 
\item ggvis has a similar underlying theory to ggplot2 (the grammar of graphics), but it’s expressed a little differently, and adds new features to make your plots interactive.
\end{itemize}
\textit{(Quotes from ggvis website)}
\end{frame}

%==========================================================================%

\begin{frame}
	\frametitle{Data Visualization with ggvis}
	\Large
\vspace{-1cm}
	\textbf{Introduction}
	\begin{itemize}
\item  ggvis also incorporates Shiny’s \textit{reactive programming} model and dplyr’s \textit{grammar of data transformation}.
\end{itemize}
\textit{(Quotes from ggvis site)}
\end{frame}
\end{document}
%==========================================================================%
\begin{frame}[fragile]
	\frametitle{Data Visualization with ggvis}
	\Large
	
	\begin{itemize}
\item The graphics produced by ggvis are fundamentally \textbf{web graphics} and work very differently from traditional \texttt{R} graphics.
\item This allows us to implement exciting new features like interactivity, but it comes at a cost. 

\item For example, every interactive ggvis plot must be connected to a running R session 
\item Static plots do not need a running R session to be viewed. 
	\end{itemize}
\textit{(Quotes from ggvis site)}
\end{frame}
%==========================================================================%
\begin{frame}[fragile]
	\frametitle{Data Visualization with ggvis}
	\Large
\begin{itemize}
\item This is great for exploration, because you can do anything in your interactive plot you can do in \texttt{R}, but it’s not so great for publication. 
\item We will overcome these issues in time, but for now be aware that we have many existing tools to reimplement before you can do everything with \textbf{ggvis} that you can do with base graphics.
\end{itemize}
\textit{(Quotes from ggvis site)}

\end{frame}

%==========================================================================%
% % CUT THIS SLIDE
\begin{frame}[fragile]
\frametitle{Data Visualization with ggvis}
\Large

The \textbf{\textit{ggvis}} vignette is divided into four main sections:
\begin{enumerate}
\item Dive into plotting with \texttt{ggvis()}.
\item Add interactivity with the mouse and keyboard.
\item Create more types of graphic by controlling the layer type.
\item Build up rich graphics with multiple layers.
\end{enumerate}
% Each section will introduce to a major idea in ggvis, and point to more detailed explanation in other vignettes.
\end{frame}


%==========================================================================%
\begin{frame}[fragile]
\frametitle{Data Visualization with ggvis}
\Large
\textbf{ggvis()}
\begin{itemize}
\item Every ggvis graphic starts with a call to \texttt{ggvis()}. 
\item The first argument is the data set that you want to plot.
\item Then the other arguments that describe how to map variables to visual properties.
\end{itemize}
\begin{framed}
\begin{verbatim}
p <- ggvis(mtcars, x = ~wt, y = ~mpg)
\end{verbatim}
\end{framed}
\end{frame}


%==========================================================================%
\begin{frame}[fragile]
	\frametitle{Data Visualization with ggvis}
	\Large
	\textbf{ggvis()}
\begin{itemize}
\item This doesn’t actually plot anything because you haven’t told ggvis how to display your data. 
\item You do that by layering visual elements, for example with \texttt{layer\_points()}
\end{itemize}

\end{frame}
%===================================================================================== %

\end{document}