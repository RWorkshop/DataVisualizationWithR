\documentclass[MASTER.tex]{subfiles} 
\begin{document} 
	
%=================================================%
% COMMON PLOT FUNCTIONS

%=================================================%
\begin{frame}[fragile]
\frametitle{Controlling axis and legends}
\Large
\begin{itemize}
\item We can control the axes using the \texttt{add\_axis}
function
\item This controls axis labels, tick marks and even grid
lines
\end{itemize}

\begin{framed}
	\begin{verbatim}
add_axis("x", title = "Month")
\end{verbatim}
\end{framed}

\end{frame}
%=================================================%
%=================================================%
\begin{frame}[fragile]
\textbf{Controlling axis and legends}
\Large

\texttt{The add\_legend} and \texttt{hide\_legend} functions allow
us to control if we see a legend and where it
appears

\begin{framed}
\begin{verbatim}
hide_legend("fill")
add_legend(c("fill", shape))
\end{verbatim}
\end{framed}
\end{frame}
%=================================================%
%=================================================%
\begin{frame}[fragile]
\frametitle{Scales}
\Large
ggvis has fewer scale functions than in ggplot2
but control much more

\begin{framed}
\begin{verbatim}
> grep("^scale", objects("package:ggvis"), value = TRUE)
[1] "scale_datetime" "scale_logical" "scale_nominal" "scale_numeric"
[5] "scale_ordinal" "scale_singular" "scaled_value" 
\end{verbatim}
\end{framed}

\end{frame}
%=================================================%

%=================================================%
\begin{frame}[fragile]
\frametitle{ggvis VS ggplot2 : How are they similar?}
\Large
\begin{itemize}
\item We can layer graphics in a similar fashion
\item Aesthetics can be set based on variables in the
data
\item We can control the type of plot with specific
functions
\end{itemize}
\end{frame}
%=================================================%
\begin{frame}
	\Large
\textbf{\textit{ggvis} vs \textit{ggplot2} : How are they different?} \\ \bigskip

From point of view of \textbf{\textit{ggvis}}
\begin{itemize}
\item Only one main plot function to work with as
opposed to two
\item Layering is done using $ \%>\% $ rather than +
\item Fewer scale functions
\item Much functionality is not yet available in \textbf{\textit{ggvis}} e.g.
facetting
\end{itemize}

\end{frame}
%=================================================%
\begin{frame}
	\Large
\textbf{Which should I use?}
\begin{itemize}
\item For static graphics: \textbf{\textit{ggplot2}}
\item For interactive graphics: \textbf{\textit{ggvis}}
\item \textbf{WARNING:} If you are using ggvis remember it's still being actively
developed and may change in structure and functionality
\end{itemize}
\end{frame}
%=================================================%
\end{document}
	