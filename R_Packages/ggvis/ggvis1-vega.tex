
%=================================================%
%% What is Vega?



\documentclass[MASTER.tex]{subfiles} 
\begin{document} 
%=================================================%
\begin{frame}
	\frametitle{Viewing ggvis graphics}
	\large
	\begin{itemize}
		\item ggvis uses \textbf{Vega} to render graphics in a web
		browser
		\item In RStudio the default it to use the "Viewer" pane
		\item From the web browser we can download SVG or
		png version of our graphics 
	\end{itemize}
\end{frame}
%======================================================================== %
\begin{frame}
\frametitle{ggvis and Vega/D3}
\begin{itemize}
\item While ggvis is built on top of \textbf{vega}, which in turn borrows many ideas from \textbf{d3}, it is designed more for data 
exploration than data presentation. 
\item This means that ggvis makes many more assumptions about what you’re trying to do: this allows it to be much more concise, at some cost of generality.
\end{itemize}
\end{frame}
%======================================================================== %
\begin{frame}
	\frametitle{ggvis and Vega/D3}
The main difference to vega is that ggvis provides a tree like structure allowing properties and data to 
be specified once and then inherited by children.
\end{frame}
%==========================================================================%
\begin{frame}
\frametitle{ggvis and Vega/D3}
\begin{itemize}
\item Vega plays a similar role to ggvis that grid does to ggplot2. That means that you shouldn’t have to know anything 
about vega to use ggvis effectively, and you shouldn’t have to refer to the vega docs to solve common problems. 
\item However, some knowledge of how vega works is likely to be necessary when you start doing more complex layouts or 
when you start pushing the limits of the ggvis DSL.
\end{itemize}
\end{frame}
%==========================================================================%
\end{document}