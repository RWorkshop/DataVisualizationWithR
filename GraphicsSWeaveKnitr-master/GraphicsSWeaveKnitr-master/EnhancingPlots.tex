\subsection{Changing your plot character}

To change the plot character (the symbol for each covariate, we supply an additional argument to the plot() function.  This argument is formulated as pch=n where n is some number.
Additionally we change the colour of the characters, by specifying a colour in the col argument.
\footnotesize \begin{verbatim}
plot(x,y,pch=15,col="red")		#Square plot symbols
plot(x,y,pch=16,col="green")		#Orb plot symbols
plot(x,y,pch=17,col="mauve")		#Triangular plot symbols
plot(x,y,pch=36	,col="amber")		#Dollar sign plot symbols
\end{verbatim}\normalsize
Recall that we can add new variates to an existing scatterplot using the points() function. Remember to set the vertical and horizontal limits accordingly.
\footnotesize \begin{verbatim}
y1 = rnorm(10); y2 = rnorm(10)
plot(x,y1, pch=8,col="purple" ,xlim=c(-5,5),ylim=c(-5,5))
points(x,y2,pch=12,col="green")
\end{verbatim}\normalsize
\subsection{Adding the regression model line}

The \texttt{abline()} function can be used to add a regression model line  by supplying as an argument the \texttt{coef()} values for intercept and slope estimates .These estimates can be inputted directly by using both functions in conjunction.

\footnotesize \begin{verbatim}
Fit1 =lm(y1~x);  coef(Fit1)
abline(coef(Fit1))	
\end{verbatim}\normalsize

\subsection{Adding a title }

It is good practice to label your scatterplots properly. You can specify the following argument
\begin{itemize}
\item	main="Scatterplot Example", 	This provides the plot with a title
\item	sub="Subtitle",                 This adds a subtitle
\item	xlab="X variable ",				This command labels the x axis
\item   ylab="y variable ",				This command labels the y-axis
\end{itemize}
We can also add text to each margin, using the \texttt{mtext()} command.
We simply require the number of the side. (1 = bottom, 2=left,3=top,4=right).
We can change the colour using the col argument.
\footnotesize \begin{verbatim}
plot(x,y,main="Scatterplot Example",   sub="subtitle",    xlab="X variable ", ylab="y variable ")	
mtext("Enhanced Scatterplot", side=4,col="red ")
\end{verbatim}\normalsize
Alternatively , we can also use the command title() to add a title to an existing scatterplot.
\footnotesize \begin{verbatim}
title(main="Scatterplot Example)	
\end{verbatim}\normalsize


\section{Combining plots}
It is possible to combine two plots. We used the graphical parameters command \texttt{par()} to create an array.
Often we just require two plots side by side or above and below. We simply specify the numbers of rows and columns of this array using the \texttt{mfrow} argument, passed as a vector.

\begin{verbatim}
par(mfrow=c(1,2))
plot(x,y1)			# draw first plot
plot(x,y2)			# draw second plot
par(mfrow=c(1,1))		# reset to default setting.
\end{verbatim}




\section{Plot of single vectors}
If only one vector is specified i.e. \texttt{plot(x)},  the plot created will simply be a scatter-plot of the values of x against their indices.

$plot(x)$
Suppose we wish to examine a trend that these points represent. We can connect each covariate using a line.

$plot(x, type = "l")$
If we wish to have both lines and points, we would input the following code. This is quite useful if we wish to see how a trend develops over time.
$plot(x, type = "b")$


