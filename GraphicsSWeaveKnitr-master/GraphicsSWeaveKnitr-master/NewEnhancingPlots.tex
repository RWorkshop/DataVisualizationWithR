Section 1 Enhancing your scatter plots
1)	Adding lines
Previously we have used scatter plots to plot bivariate data. They were constructed using the plot() command.  
Recall that we can use the arguments xlim and ylim to control the vertical and horizontal range of the plots, by specifying a two element vector (min and max) for each.
Using the abline() command, we can add lines to our scatter plots. We specify the argument according to the type of line required. A demonstration of three types of line is provided below.
Additionally we change the colour of the added lines, by specifying a colour in the col argument. We can also change the line type to one of four possible types, using the lty argument.
The line types are follows
 
1)	lty =1   Normal full line (default)
2)	lty =2   Dashed line 
3)	lty =3   Dotted line
4)	lty =4   Dash-dot line
 
x=rnorm(10)
y=rnorm(10)
plot(x,y)
plot(x,y,xlim=c(-4,4),ylim=c(-4,4))
abline(v =0 , lty =2 )    # add a vertical dotted line (here the y-axis) to the plot
abline(h=0  ,lty =3)    # add a horizontal dotted line (here the x-axis) to the plot
abline(a=0,b=1,col="green") # add a line to your plot with intercept “a” and slope “b”

2)	Changing your plot character.

To change the plot character (the symbol for each covariate, we supply an additional argument to the plot() function.  This argument is formulated as pch=n where n is some number.
Additionally we change the colour of the characters, by specifying a colour in the col argument.
plot(x,y,pch=15,col="red")		#Square plot symbols
plot(x,y,pch=16,col="green")		#Orb plot symbols
plot(x,y,pch=17,col="mauve")		#Triangular plot symbols
plot(x,y,pch=36	,col="amber")		#Dollar sign plot symbols
Recall that we can add new variates to an existing scatterplot using the points() function. Remember to set the vertical and horizontal limits accordingly.
y1 = rnorm(10); y2 = rnorm(10)
plot(x,y1, pch=8,col="purple" ,xlim=c(-5,5),ylim=c(-5,5))
points(x,y2,pch=12,col="green")

3)	Adding the regression model line to a scatterplot

The abline() function can be used to add a regression model line  by supplying as an argument the coef() values for intercept and slope estimates .These estimates can be inputted directly by using both functions in conjunction.

Fit1 =lm(y1~x);  coef(Fit1)
abline(coef(Fit1))	


4)	Adding a title to your plot
It is good practice to lable your scatterplots properly. You can specify the following argument
1)	main="Scatterplot Example", 		This provides the plot with a title
2)	sub="Subtitle",
3)	xlab="X variable ",				The gives the x axis a label
4)	ylab="y variable ",				This gives the y-axis a label
We can also add text to each margin, using the mtext() command.  We simply require the number of the side. (1 = bottom, 2=left,3=top,4=right). We can change the colour using the col argument.
plot(x,y,main="Scatterplot Example",   sub="subtitle",    xlab="X variable ", ylab="y variable ")	
mtext("Enhanced Scatterplot", side=4,col="red ")
Alternatively , we can also use the command title() to add a title to an existing scatterplot.
title(main="Scatterplot Example)	



5)	Combining plots
It is possible to combine two plots. We used the graphical parameters command par() to create an array of times. Often we just require two plots side by side or above and below. We simply specify the numbers of rows  and  columns of this array using the mfrow argument, passed as a vector.

par(mfrow=c(1,2))
plot(x,y1)			# draw first plot
plot(x,y2)			# draw second plot
par(mfrow=c(1,1))		# reset to default setting.
      
6)	Plot of single vectors.
If only one vector is specified i.e. plot(x),  the plot created will simply be a scatter-plot of the values of x against their indices.
plot(x)
Suppose we wish to examine a trend that these points represent. We can connect each covariate using a line.
plot(x, type = "l")
If we wish to have both lines and points, we would input the following code. This is quite useful if we wish to see how a trend develops over time.
plot(x, type = "b")

